\section{章节结构测试}这节用来展示文章的5层结构。事实上,一般来说文章层次在3-4层为宜。在之后的section中,我们会只使用至多3层结构(即,节-小节-子节)来进行各种演示。
 

\subsection{小节标题}这一小节我们介绍这些内容。


\subsubsection{子节标题}这一子节我们介绍这些内容。


\paragraph{段标题}这一段我们介绍这些内容。

\section{注释——脚注}
\subsection{脚注}
\par 这里是脚注测试\footnote{1111111111}这里是脚注测试这里是脚注测试这里是脚注测试\footnote{2222222222}这里是脚注测试这里是脚注测试这里是脚注测试这里是脚注测试这里是脚注测试这里是脚注测试这里是脚注测试这里是脚注测试这里是脚注测试这里是脚注测试这里是脚注测试这里是脚注测试这里是脚注测试这里是脚注测试这里是脚注测试\footnote{3333333333}这里是脚注测试这里是脚注测试这里是脚注测试这里是脚注测试这里是脚注测试这里是脚注测试这里是脚注测试这里是脚注测试这里是脚注测试这里是脚注测试这里是脚注测试这里是脚注测试

\textbf{\uline{注意!正如这份演示中所出现的情况,若该页(也就是本文档中的前一页)剩余空间不大,不足以显示足够多的文档与脚注,那么该段文字就会被移至下一页而留下空白。目前我们尚未找到解决的方法,所以如果遇到了这个问题,请修改排版,以留下足够大的空间。}}



\section{文献引用的演示}

\par 本模板使用两种方式进行参考文献导入:

\par 一种是使用biblatex进行文献管理,这是一套相对较新的系统。另外,使用了hushidong制作的符合gb7714-2015标准的biblatex样式。在此对他的工作表示感谢,要完成这样的样式非常不容易。本模板中gb7714-2015.bbx与gb7714-2015.cbx即为他的作品,在这里打包发布以便使用。

\par 首先,你要将你论文中需要引用的文献将它放到BibTeX文献库中,即bib后缀的文件,默认的bib文件位于~/reference/thesis-ref.bib,BibTeX格式在百度学术与谷歌学术可以直接导出,复制到这个文献库中,然后你可以在正文中引用,引用过的文献会自动添加到后面的参考文献中。


\par 文献\parencite{Yang_Hy200215}中提到xxxxxxx。  %/parencite后大括号中填的是BibTeX文献库中需要引用的文献第一个字段,即AUTHOR=“”前的一个字段
\par 文献\parencite{Joa1999}中提到yyyyyyy。
\par 文献\parencite{Altman1997}中提到zzzzzzz。
\par \textcolor{blue}{\textbf{\uline{本模板使用parencite而不是cite命令,因为这样能与脚注所产生编号进行区分。当然,如果你没有脚注或尾注,那么cite命令也是推荐使用的。}}}

\par 本演示模板的bib文件,内容是由Wang Tianshu制作,在此仅作演示之用。


\par 如果你不想在文中引用每篇论文,可以使用另一种不使用BibTeX文献库的方式,不过我没有找到修改这种方法参考文献标题格式的方法,如果您知道,也可以告诉我,感谢

