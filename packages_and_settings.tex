%不要改动,除非你知道自己在做什么
%不要改动,除非你知道自己在做什么
%不要改动,除非你知道自己在做什么
\usepackage[thmmarks,hyperref]{ntheorem} %定义命令环境使用的宏包
\usepackage[heading,zihao=-4]{ctex} %用来提供中文支持
\usepackage{amsmath,amssymb} %数学符号等相关宏包
\usepackage{graphicx} %插入图片所需宏包
\usepackage{xspace} %提供一些好用的空格命令
\usepackage{tikz-cd} %画交换图需要的宏包
\usepackage{url} %更好的超链接显示
\usepackage{array,booktabs} %表格相关的宏包
\usepackage{caption} %实现图片的多行说明
\usepackage{float} %图片与表格的更好排版
\usepackage{ulem} %更好的下划线
\usepackage[top=2.5cm, bottom=2.0cm, left=3.0cm, right=2.0cm]{geometry} %设置页边距

\usepackage{fontspec} %设置字体需要的宏包

%设置西文字体为Times New Roman,如果没有则以开源近似字体代替
\IfFontExistsTF{Times New Roman}{
	\setmainfont{Times New Roman}
}{
	\usepackage{newtxtext,newtxmath}
}

%设置文档中文字体。优先次序:中易 > Adobe > 华文(Mac) > Fandol
\IfFontExistsTF{SimSun}{
	\setCJKmainfont[AutoFakeBold=2,ItalicFont=KaiTi]{SimSun}
}{
	\IfFontExistsTF{AdobeSongStd-Light}{
		\setCJKmainfont[AutoFakeBold=2,ItalicFont=AdobeKaitiStd-Regular]{AdobeSongStd-Light}
	}{
		\IfFontExistsTF{STSong}{
			\setCJKmainfont[AutoFakeBold=2,BoldFont=STHeiti,ItalicFont=STKaiti]{STSong}
		}{
			\setCJKmainfont[AutoFakeBold=2,ItalicFont=FandolKai-Regular]{FandolSong-Regular}
		}
	}
}
\IfFontExistsTF{SimHei}{
	\setCJKsansfont[AutoFakeBold=2]{SimHei}
}{
	\IfFontExistsTF{AdobeHeitiStd-Regular}{
		\setCJKsansfont[AutoFakeBold=2]{AdobeHeitiStd-Regular}
	}{
		\IfFontExistsTF{STHeiti}{
			\setCJKsansfont [AutoFakeBold=2]{STHeiti}
		}{
			\setCJKsansfont[AutoFakeBold=2]{FandolHei-Regular}
		}
	}
}


\IfFileExists{zhlineskip.sty}{
	%Microsoft Word 样式的1.5倍行距(按中易字体计算)
	\usepackage[
		restoremathleading=false,
		UseMSWordMultipleLineSpacing,
		MSWordLineSpacingMultiple=1.5
	]{zhlineskip}
}{
	\linespread{1.621} %1.5倍行距
}

\showboxdepth=5
\showboxbreadth=5
\setcounter{secnumdepth}{5}

%%设置标题格式
\ctexset{
	section={
		%format用于设置章节标题全局格式,作用域为标题和编号
		%字号为小四,字体为黑体,左对齐
		%+号表示在原有格式下附加格式命令
		format+ = \zihao{-4} \heiti \raggedright,
		%name用于设置章节编号前后的词语
		%前、后词语用英文状态下,分开
		%如果没有前或后词语可以不填
		name = {,、},
		%number用于设置章节编号数字输出格式
		%输出section编号为中文
		number = \chinese{section},
		%beforeskip用于设置章节标题前的垂直间距
		%ex为当前字号下字母x的高度
		%基础高度为1.0ex,可以伸展到1.2ex,也可以收缩到0.8ex
		beforeskip = 1.0ex plus 0.2ex minus .2ex,
		%afterskip用于设置章节标题后的垂直间距
		afterskip = 1.0ex plus 0.2ex minus .2ex,
		%aftername用于控制编号和标题之间的格式
		%\hspace用于增加水平间距
		aftername = \hspace{0pt}
	},
	subsection={
		format+ = \zihao{-4} \heiti ,
		%仅输出subsection编号且为中文
		number = \chinese{subsection},
		name = {(,)},
		beforeskip = 1.0ex plus 0.2ex minus .2ex,
		afterskip = 1.0ex plus 0.2ex minus .2ex,
		aftername = \hspace{0pt}
	},
	subsubsection={
		%设置对齐方式为居中对齐
		format+ = \zihao{-4} \heiti,
		indent={2em},
		%仅输出subsubsection编号,格式为阿拉伯数字,打字机字体
		number = \ttfamily\arabic{subsubsection},
		name = {,.},
		beforeskip = 1.0ex plus 0.2ex minus .2ex,
		afterskip = 1.0ex plus 0.2ex minus .2ex,
		aftername = \hspace{0pt}
	},
	paragraph={
		format+ = \zihao{-4} ,
		indent={2em},
		%格式为阿拉伯数字,打字机字体
		number = {\arabic{paragraph}},
		name = {(,)},
		beforeskip = 1.0ex plus 0.2ex minus .2ex,
		afterskip = 1.0ex plus 0.2ex minus .2ex,
		aftername = \hspace{0pt}
	}
}

\usepackage[bottom,perpage]{footmisc}               %脚注,显示在每页底部,编号按页重置
\renewcommand*{\footnotelayout}{\zihao{-5}\rmfamily}  %设置脚注为小五号宋体
\renewcommand{\thefootnote}{\textcircled{\arabic{footnote}}}    %设置脚注标记为①,②,...

%设置页眉页脚
\usepackage{fancyhdr}
\lhead{华东师范大学本科生学年论文}
\chead{}
\rhead{\Title}
\lfoot{}
\cfoot{\thepage}
\rfoot{}

\usepackage{xcolor} %彩色的文字

\usepackage[hidelinks]{hyperref} %各种超链接必备
\usepackage{cleveref} %交叉引用

%允许公式跨页显示
\allowdisplaybreaks

%屏蔽无关的Warning
\usepackage{silence}
\WarningFilter*{biblatex}{Conflicting options.\MessageBreak'eventdate=iso' requires 'seconds=true'.\MessageBreak Setting 'seconds=true'}



%使用biblatex管理文献,输出格式使用gb7714-2015标准,后端为biber
\usepackage[backend=biber,style=gb7714-2015,hyperref=true]{biblatex}
%将参考文献字体设置为五号
\renewcommand*{\bibfont}{\zihao{5}}


%表格单元格内换行
\newcommand{\tabincell}[2]{\begin{tabular}{@{}#1@{}}#2\end{tabular}}

%设置图、表的编号格式
\renewcommand{\thefigure}{\arabic{section}-\arabic{figure}}
\renewcommand{\thetable}{\arabic{section}-\arabic{table}}
%%每个section开始重置图、表的计数器
\makeatletter
\@addtoreset{table}{section}
\makeatother
\makeatletter
\@addtoreset{figure}{section}
\makeatother

%显示 1、2级标题
\setcounter{tocdepth}{2}

%设置目录字体
\usepackage{tocloft}
\renewcommand{\contentsname}{\centerline{目录}}
\renewcommand{\cftaftertoctitle}{\hfill}
\renewcommand{\cfttoctitlefont}{\sffamily \bfseries \zihao{-4}}
\renewcommand{\cftsubsubsecfont}{\rmfamily}
\renewcommand{\cftsubsecfont}{\rmfamily}
\renewcommand{\cftsecfont}{\rmfamily}
\renewcommand{\cftsecleader}{\cftdotfill{\cftdotsep}}
\renewcommand{\cftsecfont}{}
\renewcommand{\cftsecpagefont}{}

%灵活的行距定义(用于封面)
\usepackage{setspace}
%使用绝对坐标制作封面使用的宏包
\usepackage[absolute,overlay]{textpos}
  \setlength{\TPHorizModule}{1mm}
  \setlength{\TPVertModule}{1mm}